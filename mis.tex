\documentclass[12pt]{article}
\usepackage[utf8]{inputenc}
\usepackage{graphicx}
\usepackage{amsmath} % for mathematical symbols
\usepackage{amssymb} % for \mathbb command
\usepackage[backend=biber,style=numeric]{biblatex}
\graphicspath{ {images/} }
\addbibresource{references.bib} % Specify the name of your .bib file

\title{
{\textbf{Technical University of Košice}}\\
\vspace{12pt}
{\large \textbf{Faculty of Mining, Ecology, Process Control\\ and Geotechnologies}}\\
\vspace{12pt}
{\large Institute of Control and Informatization of Production Processes}\\
\vspace{64pt}
{\textbf{Mathematical method for identification, modelling and simulation}}\\
}
\author{Ing. Ales Jandera}
\date{11. July 2024}

\begin{document}

\maketitle

\newpage

\tableofcontents

\newpage

\section{Advanced mathematical methods for decision and planning}
Advanced mathematical methods for decision-making and planning are crucial
in various fields, including operations research, economics, engineering,
and artificial intelligence. These methods involve sophisticated mathematical
techniques to analyze, model, and solve complex problems. Here are some key methods:

\subsection{Linear Programming (LP)}
Linear Programming involves optimizing a linear objective function subject to
linear equality and inequality constraints~\cite{vanderbei2020linear}. The objective function represents a goal
such as maximizing profit or minimizing cost. The constraints represent limitations
or requirements, such as resource capacities, production capabilities, or budget
limits. By mathematically modeling these objectives and constraints, linear
programming helps identify the best possible outcome within the feasible region
defined by the constraints. It's widely used in resource allocation, production
planning, and logistics, providing valuable insights into how to efficiently
distribute limited resources, schedule production runs, and manage supply chains.
Additionally, linear programming can be applied to areas like transportation, where
it aids in determining the most cost-effective routes for delivery, and in finance,
where it helps in portfolio optimization to maximize returns or minimize risk.\\
\\
Example of using the method:\\
Maximize\\\[ Z = c_1x_1 + c_2x_2 + \ldots + c_nx_n \]\\
Subject to:\\
\[ a_{11}x_1 + a_{12}x_2 + \ldots + a_{1n}x_n \leq b_1 \]
\[ a_{21}x_1 + a_{22}x_2 + \ldots + a_{2n}x_n \leq b_2 \]
\[ \vdots \]
\[ a_{m1}x_1 + a_{m2}x_2 + \ldots + a_{mn}x_n \leq b_m \]
\[ x_i \geq 0 \text{ for all } i \]

\subsection{Dynamic Programming (DP)}
Dynamic Programming solves complex problems by breaking them down into simpler
subproblems, addressing each subproblem only once and storing its solution~\cite{bellman1957dynamic}.
This approach avoids the redundant computations typical of naive recursive methods.
Dynamic Programming is particularly useful for problems with overlapping subproblems,
where the same subproblems arise multiple times, and optimal substructure properties,
where the optimal solution to the problem can be constructed from optimal solutions
to its subproblems. Applications of Dynamic Programming span various fields,
including computer science for algorithm design, operations research for resource
management, economics for decision-making, and bioinformatics for sequence alignment.
Classic examples include the Fibonacci sequence, shortest path problems like
Dijkstra's algorithm, knapsack problems, and dynamic programming on trees.
By structuring these problems into a systematic framework, Dynamic Programming
provides efficient and scalable solutions, making it a powerful tool in both
theoretical and applied contexts.\\
\\
Example of using the method:\\
\\
Fibonacci sequence calculation:\\
\[ F(n) = F(n-1) + F(n-2) \]
With base cases:\\
\[ F(0) = 0, F(1) = 1 \]

\subsection{Markov Decision Processes (MDPs)}
Decision-making in situations where outcomes are partly random and partly under
the control of a decision maker~\cite{puterman1994markov}.
An MDP is defined by its states, actions, transition probabilities, and rewards.
The decision maker, or agent, chooses actions based on the current state, and
transitions to a new state according to the transition probabilities, receiving a
reward associated with the action taken. The goal is to find a policy, a mapping
from states to actions, that maximizes the expected cumulative reward over time.
MDPs are widely used in robotics for path planning and control, in economics for
modeling decision-making under uncertainty, and in artificial intelligence,
particularly in reinforcement learning, where agents learn optimal policies through
interactions with the environment. The framework of MDPs allows for the formalization
and solution of a wide range of sequential decision-making problems, making them a
fundamental tool in both theoretical research and practical applications.\\

Consists of:
\begin{itemize}
    \item States \( S \)
    \item Actions \( A \)
    \item Transition probabilities \( P(s'|s,a) \)
    \item Rewards \( R(s,a) \)
    \item Policy \( \pi(s) \)
\end{itemize}

\subsection{Game Theory}
Game Theory studies strategic interactions between decision-makers, where the outcome
for each participant or player depends on the actions of all involved~\cite{osborne1994course}. It provides a
framework for understanding and analyzing situations where players make decisions
that are interdependent, leading to outcomes that depend on the choices of others.
Game theory covers a variety of games, including cooperative and non-cooperative
games, zero-sum and non-zero-sum games, and games of complete and incomplete
information. It is used in economics to model market behavior, in political science
to analyze voting systems and strategy, and in evolutionary biology to study the
evolution of cooperation and competition among species. Game theory also finds
applications in areas like computer science for algorithm design, network security,
and artificial intelligence, particularly in multi-agent systems. Concepts such as
Nash equilibrium, dominant strategies, and Pareto efficiency are central to game
theory, providing insights into how rational players would behave in strategic
situations, and guiding the design of mechanisms and systems for optimal
decision-making.\\
\\
Example:\\
Prisoner's Dilemma, where two players can either cooperate or defect.
Both cooperate: each gets 3 points. Both defect: each gets 1 point.
One cooperates, the other defects: cooperator gets 0, defector gets 5.

\subsection{Bayesian Decision Theory}
Bayesian Decision Theory incorporates probabilistic models to make decisions, providing a rigorous
framework for statistical decision-making and machine learning~\cite{berger1985statistical}. It leverages Bayes'
theorem to update the probability estimates of outcomes based on new evidence.
The key components of Bayesian Decision Theory include prior probabilities, which
represent initial beliefs about the likelihood of outcomes; likelihoods, which describe
the probability of observing the evidence given an outcome; and posterior probabilities,
which are updated beliefs after considering the evidence. Decision rules are then used
to select the action that maximizes expected utility or minimizes expected loss based
on these posterior probabilities. This theory is foundational in many areas, including
medical diagnosis, where it helps in assessing the probability of diseases given
symptoms; finance, for updating investment strategies based on market data; and machine 
learning, particularly in Bayesian inference methods, where models are continuously
refined as more data becomes available. By incorporating uncertainty and learning from
data, Bayesian Decision Theory provides a powerful approach to making informed and
rational decisions in uncertain environments. Components: Prior probabilities, Likelihoods,
Posterior probabilities, Decision rules.

\subsection{Reinforcement Learning (RL)}
Reinforcement Learning involves agents learning to make decisions by interacting with an
environment~\cite{sutton2018reinforcement}. The agent takes actions in the environment and receives feedback in the
form of rewards or penalties. The goal of the agent is to learn a policy, which is a
mapping from states to actions, that maximizes the cumulative reward over time.
Reinforcement Learning is a key area in artificial intelligence and robotics, where it
is used to develop systems that can adapt and improve their performance through
experience. This approach is based on the principles of trial and error, and involves
exploring the environment to discover which actions yield the highest rewards.
Key concepts in reinforcement learning include the value function, which estimates
the expected reward for each state; the Q-function, which estimates the expected
reward for state-action pairs; and algorithms such as Q-learning, policy gradients,
and deep reinforcement learning, which combine reinforcement learning with deep neural
networks. Applications of reinforcement learning span various domains, including
autonomous vehicles, game playing (e.g., AlphaGo), robotic control, and personalized
recommendations, showcasing its versatility and power in solving complex decision-making
problems.\\
\\
Q-Learning algorithm:
\[ Q(s,a) \leftarrow Q(s,a) + \alpha [r + \gamma \max_{a'} Q(s',a') - Q(s,a)] \]
%
%
\section{Advanced methods in Graph Theory}
%
    Graph theory is a fundamental field of mathematics and computer science that
    studies the properties and applications of graphs, which consist of vertices
    connected by edges. It is essential for solving problems in various domains like
    \textbf{telecommunications} used for optimizing the routing of data to maximize
    throughput. \textbf{Transportation and Supply Chain Management} determining the
    most efficient way to route goods and services. \textbf{Utility Networks} to
    managing water, gas, or electricity distribution to maximize efficiency and
    reliability. \textbf{Project Management} using graph theory to task scheduling
    where tasks are dependent on the completion of preceding tasks.
    \textbf{Epidemiology} modeling the spread of diseases through dynamically
    changing human contact networks.
%
    \subsection{Network flow}
    Network flow is a fundamental concept in graph theory that models the
    flow of quantities through a network, allowing for the analysis and
    optimization of systems.

    \subsubsection{Basic Components of Network Flow}

    For our purpose as in operations research, we will use terminologyu as a directed
    graph is called a network, the vertices are called nodes and the edges are called arcs.
    \textbf{Directed Graph} used for a network flow problem where each edge has
    a capacity and each node is either a source, a sink, or an intermediate node.
    Nodes represent junction points, and edges represent paths that the flow can take.
    Source are the origin node from where the flow starts and the sink is the
    destination node~\cite{diestel2017graph} where the flow is intended to end. Each edge in the network has
    an associated capacity, which is the maximum flow that the edge can handle.
    
    \textbf{Flow} is the amount of stuff (e.g., data, goods, traffic) that passes along the edges.
    It must satisfy \textbf{Capacity Constraint condition} that flow on an edge cannot
    exceed the capacity of the edge and 
    \textbf{Conservation of Flow condition} except for the source and sink, the flow
    into any node must equal the flow out of that node.

    \subsubsection{Key Concepts}

    \textbf{Maximum Flow} is the main problem in network flow to determine the
    maximum flow from the source to the sink without exceeding the capacities
    of the edges. \textbf{Flow Value} is the total amount of flow that leaves
    the source or enters the sink (these two values are equal due to the
    conservation of flow).
    
    \textbf{Residual Network} is constructed from the original flow network by
    including residual capacities which indicate additional possible flow on an
    edge given the current flow. \textbf{Augmenting Path} in the residual network
    is a path from the source to the sink along which additional flow can be pushed
    to increase the overall flow in the network.
    
    \textbf{Cut} in a network is a partition of the vertices into two disjoint subsets
    that separate the source from the sink. The capacity of the cut is the sum of
    the capacities of the edges that are directed from the subset containing the
    source to the subset containing the sink. The max-flow min-cut theorem states
    that the maximum value of an source to sink flow is equal to the minimum
    capacity of an same cut in the network.
    
    \subsection{Graph Coloring}
    Graph coloring is a fundamental concept in graph theory, which involves assigning
    colors to elements of a graph under certain constraints. The most common form
    of graph coloring is vertex coloring, where colors are assigned to vertices
    such that no two adjacent vertices has the same color. This concept
    is not only pivotal in theoretical mathematics but also has practical
    applications in areas such as scheduling, networking, and the allocation
    of resources.

    \subsubsection{Types of Graph Coloring}\label{color}

    The minimum number of colors needed to achieve \textbf{Vertex Coloring} is
    known as the graph's chromatic number.\\ Proper vertex coloring is a concept
    in graph theory where the vertices of a graph are colored in such a way that
    no two adjacent vertices share the same color. For a proper vertex coloring
    using \( k \) colors in a graph \( G \) with vertex set \( V \), the coloring
    function \( c \) can be described as:
    \begin{equation}
        c: V \rightarrow \{1, 2, ..., k\}
    \end{equation}
    Such that for every edge \( (u, v) \in E \) (the edge set of \( G \)):
    \begin{equation}
        c(u) \neq c(v)
    \end{equation}
    \textbf{Edge Coloring} incident similar to vertex coloring but applies to edges.
    Here, no two edges sharing a common vertex, can have the same color.
    The minimum number of colors needed for an edge coloring of a graph is called the
    chromatic index \( \chi'(G) \). For a graph \( G = (V, E) \) with vertex
    set \( V \) and edge set \( E \), and a given number of colors \( k \), an edge
    coloring is a function:
    \begin{equation}
        c: E \rightarrow \{1, 2, \dots, k\}
    \end{equation}
    Such that for any two edges \( e_1 \) and \( e_2 \) that share a common vertex,
    the following must hold:
    \begin{equation}
        c(e_1) \neq c(e_2)
    \end{equation}
    \\
    \textbf{Face Coloring} is used in planar graphs where faces (the regions
    bounded by edges) must be colored in such a way that no two faces sharing a
    boundary segment have the same color. This is directly related to the
    Four Color Theorem~\ref{fct}, which states that four colors are sufficient to color
    any planar graph.

    \subsubsection{Key Concepts and Theorems}

    \textbf{Four Color Theorem}\label{fct} is one of the most famous theorems in
    graph theory, it states that four colors are sufficient to color the vertices
    of any planar graph so that no two adjacent vertices use the same color.
    A planar graph is a graph that can be embedded in the plane, meaning it can
    be drawn on a flat surface without any of its edges crossing each other
    For any planar graph \( G \) can be phrased as:
    \begin{equation}
        \chi(G) \leq 4
    \end{equation}
    \\
    \textbf{Brooks' Theorem} provides a bound on the chromatic number of a graph.
    It states that any connected graph other than a complete graph or an odd cycle
    has chromatic number at most $\Delta$ (the maximum degree of the graph).
    
    \textbf{Hall's Marriage Theorem} often used in edge coloring and matching problems, it provides a condition
    that guarantees the existence of a perfect matching in bipartite graphs,
    which is directly related to the graph coloring problem~\ref{color}.
    It states that a bipartite graph has a perfect matching if and only if for every
    subset \( S \) of one part of the graph, the number of neighbors of \( S \)
    (denoted as \( N(S) \)) is at least as large as \( S \):
    \begin{equation}
            |N(S)| \geq |S|
    \end{equation}

    \subsubsection{Algorithms for Graph Coloring}

    \textbf{Greedy Coloring} is a straightforward and easy-to-implement method where
    vertices are colored one at a time, using the smallest available color that
    hasn’t been used by adjacent vertices. This method does not generally yield
    the minimum number of colors needed. The greedy algorithm attempts to color
    each vertex with the lowest number \( k \) where \( k \) is the smallest
    positive integer not used by its adjacent vertices. If \( d(v) \) is the
    degree of vertex \( v \), then a vertex can be colored with one
    of \( d(v) + 1 \) colors:
    \begin{equation}
        c(v) = \min \{ k \in \mathbb{N} \mid k \notin \{c(u) \mid u \text{ is adjacent to } v\}\}
    \end{equation}
    \\
    \textbf{Backtracking Algorithm} is a more exhaustive approach that can find the
    chromatic number of a graph but may be computationally expensive for large graphs.
    Formally, it's defined by:
    \begin{equation}
        \chi(G) = \min \{k \mid G \text{ is k-colorable} \}
    \end{equation}
    \\
    \textbf{Heuristic Algorithms} solve many practical problems use heuristic or
    approximation algorithms to find a "good-enough" coloring. Examples include
    the DSATUR algorithm, which orders vertices based on their degree and
    saturation.

    \subsection{Dynamic Graphs}

    Dynamic graphs are an extension of traditional graph theory that deals with
    graphs in which the structure changes over time. These changes can include the
    adding or removing of vertices and edges, as well as changes in the weights
    or labels of existing vertices and edges. Dynamic graphs are particularly
    relevant in fields where the basic relationships between entities are
    not static but evolve, such as social networks, traffic networks,
    telecommunications, and biological networks.

    \subsubsection{Key Concepts}

    \textbf{Temporal Dimension} use time as a fundamental component,
    allowing for the exploration of how graph properties evolve. This temporal
    dimension distinguishes dynamic graphs from static graphs, where the
    structure is fixed.
    
    Look at three \textbf{Types of Changes} in dynamic graphs. \textbf{Vertex Dynamics}
    includes the adding or removing of nodes. New users joining a
    social network or stations being added to a transport network.
    \textbf{Edge Dynamics} involves the adding or removing of edges, which could
    represent forming or dissolving relationships in a social network or route changes in
    transport system. \textbf{Attribute Changes} the properties and weights of vertices and edges,
    such as changing bandwidth in communication networks or fluctuating relationship strengths in social networks.\\
    Discrete Time Models changes are modeled at specific time steps or intervals in oposite
    Continuous Time Models evolves the graph continuously over time, and changes can
    occur at any moment.

    \subsubsection{Analytical Challenges}

    The dynamic nature adds a layer of \textbf{complexity} to traditional graph
    problems, making them computationally harder to solve.\\
    Keeping track of changes over time requires \textbf{efficient memory management}
    and computational strategies, especially for large-scale networks.\\
    \textbf{Predicting} how a graph will evolve involves understanding past trends
    and potentially complex dependencies between changes.

    \subsubsection{Problems and Algorithms in Dynamic Graphs}
    \textbf{Dynamic Connectivity} determines whether two vertices are connected by a path and how this
    connectivity changes over time. Efficient algorithms are required to update
    connectivity information as the graph evolves.
    
    \textbf{Dynamic Shortest Paths} is try to find the shortest paths in a graph that changes over time is crucial
    for applications like dynamic routing in transportation and communication
    networks. The problem of maintaining shortest paths in a dynamic graph, where edges
    can be added, removed, or have their weights changed, can be described as follows:\\
    \\
    Distance Update After Edge Weight Change:
    \begin{equation}
        d(u, v) = \min(d(u, v), d(u, k) + w(k, v)),
    \end{equation}
    where, \( d(u, v) \) represents the shortest path from \( u \) to \( v \),
    and \( w(k, v) \) represents the new weight of an edge that has been updated.
    
    \textbf{Community Detection} in social and biological networks, identifying groups or communities that
    change over time can provide insights into the underlying structure and
    dynamics of the network. The modularity \( Q \) at any time \( t \) can be
    formulated as:
    \begin{equation}
        Q_t = \sum_{c \in C} \left[ \frac{L_c}{L} - \left( \frac{d_c}{2L} \right)^2 \right],
    \end{equation}
    where \( L_c \) is the number of edges within community \( c \), \( d_c \) is the
    sum of degrees of the nodes in \( c \), and \( L \) is the total number of edges
    in the graph.
    
    \textbf{Network Resilience}
    analyzing how structural changes affect the stability and resilience of
    networks against failures or attacks.\\
    Assessing network resilience in dynamic graphs might involve calculating the
    algebraic connectivity, represented by the Fiedler value (the second smallest
    eigenvalue of the Laplacian matrix), which can be influenced by dynamic changes:
    \begin{equation}
        L = D - A,
    \end{equation}
    where \( D \) is the degree matrix and \( A \) is the adjacency matrix of the
    graph.\\
    Algebraic Connectivity (Fiedler Value):
    \begin{equation}
        \lambda_2 = \min_{x \neq 0} \frac{x^T L x}{x^T x}
    \end{equation}

\newpage

\printbibliography

\end{document}