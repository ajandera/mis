\documentclass[12pt]{article}
\usepackage[utf8]{inputenc}
\usepackage{graphicx}
\usepackage{amsmath} % for mathematical symbols
\usepackage{amssymb} % for \mathbb command
\usepackage[backend=biber,style=numeric]{biblatex}
\graphicspath{ {images/} }
\addbibresource{references.bib} % Specify the name of your .bib file

\title{
{\textbf{Technical University of Košice}}\\
\vspace{12pt}
{\large \textbf{Faculty of Mining, Ecology, Process Control\\ and Geotechnologies}}\\
\vspace{12pt}
{\large Institute of Control and Informatization of Production Processes}\\
\vspace{64pt}
{\textbf{Mathematical method for identification, modelling and simulation}}\\
}
\author{Ing. Ales Jandera}
\date{11. May 2024}

\begin{document}

\maketitle

\newpage

\tableofcontents

\newpage

\section{XXX}
%
%
\section{Advanced methods in Graph Theory}
%
    Graph theory is a fundamental field of mathematics and computer science that
    studies the properties and applications of graphs, which consist of vertices
    connected by edges. It is essential for solving problems in various domains like
    \textbf{telecommunications} used for optimizing the routing of data to maximize
    throughput. \textbf{Transportation and Supply Chain Management} determining the
    most efficient way to route goods and services. \textbf{Utility Networks} to
    managing water, gas, or electricity distribution to maximize efficiency and
    reliability. \textbf{Project Management} using graph theory to task scheduling
    where tasks are dependent on the completion of preceding tasks.
    \textbf{Epidemiology} modeling the spread of diseases through dynamically
    changing human contact networks.
%
    \subsection{Network flow}
    Network flow is a fundamental concept in graph theory that models the
    flow of quantities through a network, allowing for the analysis and
    optimization of systems.

    \subsubsection{Basic Components of Network Flow}

    \textbf{Directed Graph} used for a network flow problem where each edge has
    a capacity and each node is either a source, a sink, or an intermediate node.
    Nodes represent junction points, and edges represent paths that the flow can take.
    Source are the origin node from where the flow starts and th sink is the
    destination node where the flow is intended to end. Each edge in the network has
    an associated capacity, which is the maximum flow that the edge can handle.\\
    \\
    \textbf{Flow} is the amount of stuff (e.g., data, goods, traffic) that passes along the edges.
    It must satisfy \textbf{Capacity Constraint condition} that flow on an edge cannot
    exceed the capacity of the edge and 
    \textbf{Conservation of Flow condition} except for the source and sink, the flow
    into any node must equal the flow out of that node.

    \subsubsection{Key Concepts}

    \textbf{Maximum Flow} is the main problem in network flow is to determine the
    maximum flow from the source to the sink without exceeding the capacities
    of the edges. Algorithms like the Ford-Fulkerson method, the Edmonds-Karp
    algorithm, and the Dinic's algorithm are designed to solve this problem.\\
    \\
    \textbf{Flow Value} is the total amount of flow that leaves the source or
    enters the sink (these two numbers are equal due to the conservation of flow).\\
    \\
    \textbf{Residual Network} is constructed from the original flow network by
    including residual capacities which indicate additional possible flow on an
    edge given the current flow.\\
    \\
    \textbf{Augmenting Path} in the residual network is a path from the source
    to the sink along which additional flow can be pushed to increase the overall
    flow in the network.\\
    \\
    \textbf{Cut} in a network is a partition of the vertices into two disjoint subsets
    that separate the source from the sink. The capacity of the cut is the sum of
    the capacities of the edges that are directed from the subset containing the
    source to the subset containing the sink. The max-flow min-cut theorem states
    that the maximum value of an source to sink flow is equal to the minimum
    capacity of an same cut in the network.
    
    \subsection{Graph Coloring}
    Graph coloring is a fundamental concept in graph theory, which involves assigning
    colors to elements of a graph under certain constraints. The most common form
    of graph coloring is vertex coloring, where colors are assigned to vertices
    such that no two adjacent vertices share the same color. This concept
    is not only pivotal in theoretical mathematics but also has practical
    applications in areas such as scheduling, networking, and the allocation
    of resources.

    \subsubsection{Types of Graph Coloring}\label{color}

    The goal of \textbf{Vertex Coloring} is to color the vertices of a graph such
    that no two adjacent vertices share the same color. The minimum number of colors
    needed to achieve this is known as the graph's chromatic number.\\
    For a proper vertex coloring using \( k \) colors in a graph \( G \) with
    vertex set \( V \), the coloring function \( c \) can be described as:
    \begin{equation}
        c: V \rightarrow \{1, 2, ..., k\}
    \end{equation}
    Such that for every edge \( (u, v) \in E \) (the edge set of \( G \)):
    \begin{equation}
        c(u) \neq c(v)
    \end{equation}
    \textbf{Edge Coloring} is similar to vertex coloring but applies to edges.
    Here, no two adjacent edges (edges sharing a common vertex) can have the same
    color. The minimum number of colors needed for an edge coloring of a graph is called the
    chromatic index \( \chi'(G) \). For a graph \( G = (V, E) \) with vertex
    set \( V \) and edge set \( E \), and a given number of colors \( k \), an edge
    coloring is a function:
    \begin{equation}
        c: E \rightarrow \{1, 2, \dots, k\}
    \end{equation}
    Such that for any two edges \( e_1 \) and \( e_2 \) that share a common vertex,
    the following must hold:
    \begin{equation}
        c(e_1) \neq c(e_2)
    \end{equation}
    \\
    \textbf{Face Coloring} is relevant in planar graphs where faces (the regions
    bounded by edges) must be colored in such a way that no two faces sharing a
    boundary segment have the same color. This is directly related to the
    Four Color Theorem~\ref{fct}, which states that four colors are sufficient to color
    any planar graph.

    \subsubsection{Key Concepts and Theorems}

    \textbf{Four Color Theorem}\label{fct} is one of the most famous theorems in
    graph theory, it states that four colors are sufficient to color the vertices
    of any planar graph so that no two adjacent vertices share the same color.
    For any planar graph \( G \) can be phrased as:
    \begin{equation}
        \chi(G) \leq 4
    \end{equation}
    \\
    \textbf{Brooks' Theorem} provides a bound on the chromatic number of a graph.
    It states that any connected graph other than a complete graph or an odd cycle
    has chromatic number at most $\Delta$ (the maximum degree of the graph):
    \begin{equation}
        \chi(G) \leq \Delta + 1
    \end{equation}
    However, if \( G \) is neither a complete graph nor an odd cycle:
    \begin{equation}
            \chi(G) \leq \Delta.
    \end{equation}
    \textbf{Hall's Marriage Theorem} often used in edge coloring and matching problems, it provides a condition
    that guarantees the existence of a perfect matching in bipartite graphs,
    which is directly related to the graph coloring problem~\ref{color}.
    It states that a bipartite graph has a perfect matching if and only if for every
    subset \( S \) of one part of the graph, the number of neighbors of \( S \)
    (denoted as \( N(S) \)) is at least as large as \( S \):
    \begin{equation}
            |N(S)| \geq |S|
    \end{equation}
    This theorem helps in determining if it's possible to match edges (or jobs to
    timeslots, etc.) without overlap, effectively a coloring problem on the edges.

    \subsubsection{Algorithms for Graph Coloring}

    \textbf{Greedy Coloring} is a straightforward and easy-to-implement method where
    vertices are colored one at a time, using the smallest available color that
    hasn’t been used by adjacent vertices. This method does not generally yield
    the minimum number of colors needed. The greedy algorithm attempts to color
    each vertex with the lowest number \( k \) where \( k \) is the smallest
    positive integer not used by its adjacent vertices. If \( d(v) \) is the
    degree of vertex \( v \), then a vertex can be colored with one
    of \( d(v) + 1 \) colors:
    \begin{equation}
        c(v) = \min \{ k \in \mathbb{N} \mid k \notin \{c(u) \mid u \text{ is adjacent to } v\}\}
    \end{equation}
    \\
    \textbf{Backtracking Algorithm} is a more exhaustive approach that can find the
    chromatic number of a graph but may be computationally expensive for large graphs.
    The chromatic number \( \chi(G) \) of a graph \( G \) is the smallest number
    of colors needed to color the vertices of \( G \) so that no two adjacent
    vertices share the same color. Formally, it's defined by:
    \begin{equation}
        \chi(G) = \min \{k \mid G \text{ is k-colorable} \}
    \end{equation}
    \\
    \textbf{Heuristic Algorithms} solve many practical problems use heuristic or
    approximation algorithms to find a "good-enough" coloring. Examples include
    the DSATUR algorithm, which orders vertices based on their degree and
    saturation.

    \subsection{Dynamic Graphs}

    Dynamic graphs are an extension of traditional graph theory that deals with
    graphs in which the structure changes over time. These changes can include the
    addition or deletion of vertices and edges, as well as changes in the weights
    or labels of existing vertices and edges. Dynamic graphs are particularly
    relevant in fields where the underlying relationships between entities are
    not static but evolve, such as social networks, traffic networks,
    telecommunications, and biological networks.

    \subsubsection{Key Concepts}

    \textbf{Temporal Dimension} incorporate time as a fundamental component,
    allowing for the exploration of how graph properties evolve. This temporal
    dimension distinguishes dynamic graphs from static graphs, where the
    structure is fixed.\\
    \\
    Look at three \textbf{Types of Changes} in dynamic graphs. \textbf{Vertex Dynamics}
    includes the addition or removal of nodes. Forinstance, new users joining a
    social network or stations being added to a transport network.
    \textbf{Edge Dynamics} involves the addition or removal of edges, which could
    represent forming or dissolving relationships in a social network orchanges in
    routes in a transportation system. \textbf{Attribute Changes} to the properties
    or weights of the existing nodes and edges, such as changing bandwidth in
    communication networks or fluctuating relationship strengths in social networks.\\
    Discrete Time Models changes are modeled at specific time steps or intervals in oposite
    Continuous Time Models evolves the graph continuously over time, and changes can
    occur at any moment.

    \subsubsection{Analytical Challenges}

    The dynamic nature adds a layer of \textbf{complexity} to traditional graph
    problems, making them computationally harder to solve.\\
    Keeping track of changes over time requires \textbf{efficient memory management}
    and computational strategies, especially for large-scale networks.\\
    \textbf{Predicting} how a graph will evolve involves understanding past trends
    and potentially complex dependencies between changes.

    \subsubsection{Problems and Algorithms in Dynamic Graphs}

    \textbf{Dynamic Connectivity} determines whether two vertices are connected by a path and how this
    connectivity changes over time. Efficient algorithms are required to update
    connectivity information as the graph evolves.\\
    \\
    \textbf{Dynamic Shortest Paths} finding the shortest paths in a graph that changes over time is crucial
    for applications like dynamic routing in transportation and communication
    networks. The problem of maintaining shortest paths in a dynamic graph, where edges
    can be added, removed, or have their weights changed, can be mathematically
    described as follows:\\
    \\
    Distance Update After Edge Weight Change:
    \begin{equation}
        d(u, v) = \min(d(u, v), d(u, k) + w(k, v)),
    \end{equation}
    where, \( d(u, v) \) represents the shortest distance from \( u \) to \( v \),
    and \( w(k, v) \) represents the new weight of an edge that has been updated.\\
    \\
    \textbf{Community Detection} in social and biological networks, identifying groups or communities that
    change over time can provide insights into the underlying structure and
    dynamics of the network. The modularity \( Q \) at any time \( t \) can be
    formulated as:
    \begin{equation}
        Q_t = \sum_{c \in C} \left[ \frac{L_c}{L} - \left( \frac{d_c}{2L} \right)^2 \right],
    \end{equation}
    where \( L_c \) is the number of edges within community \( c \), \( d_c \) is the
    sum of degrees of the nodes in \( c \), and \( L \) is the total number of edges
    in the graph.\\
    \\
    \textbf{Network Resilience}
    analyzing how structural changes affect the stability and resilience of
    networks against failures or attacks.\\
    Assessing network resilience in dynamic graphs might involve calculating the
    algebraic connectivity, represented by the Fiedler value (the second smallest
    eigenvalue of the Laplacian matrix), which can be influenced by dynamic changes:
    \begin{equation}
        L = D - A,
    \end{equation}
    where \( D \) is the degree matrix and \( A \) is the adjacency matrix of the
    graph.\\
    Algebraic Connectivity (Fiedler Value):
    \begin{equation}
        \lambda_2 = \min_{x \neq 0, x \perp \mathbf{1}} \frac{x^T L x}{x^T x}
    \end{equation}
    The smaller \( \lambda_2 \) is, the less connected the graph is, indicating
    lower resilience.

\newpage

\printbibliography

\end{document}